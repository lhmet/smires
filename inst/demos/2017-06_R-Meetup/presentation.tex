% !TeX program = xelatex
%%% !BIB program = biber
% !TeX spellcheck = de_at

\documentclass[xetex, german]{beamer}


\newcommand{\fullwidthgraphic}[2] {
	{
		\usebackgroundtemplate{\includegraphics[width=\paperwidth,height=\paperheight,keepaspectratio]{#1}}
		\begin{frame}
		\end{frame}
	}
}

\AtBeginDocument{
	\sisetup{
		math-micro=\text{µ},
		text-micro=µ, 
		group-minimum-digits=3,
		output-decimal-marker={,},
		per-mode = symbol}
}

\setlength{\parskip}{\medskipamount} 

\usepackage{graphicx}
\usepackage{fontspec}
\usepackage{siunitx}
\usepackage{booktabs}
\setmainfont{Open Sans}

\usepackage{polyglossia}

\usetheme{Copenhagen}
    \usecolortheme{seahorse}
    \setbeamertemplate{navigation symbols}{\usebeamerfont{footline}\insertframenumber / \inserttotalframenumber}
    \setbeamertemplate{footline}{}

\usepackage{url}
\usepackage{amsmath}


\title{smires -- Calculating Hydrological Metrics for Univariate Time Series}

\author{Tobias Gauster}

\institute
{
    Institute of Applied Statistics and Computing\\
    BOKU, Vienna 
}

\date{R Meetup, Juni 2017}


\begin{document}

    
\begin{frame}
\titlepage
\end{frame}
 
\begin{frame}{Motivation}
  \begin{itemize}
      \item SMIRES: Science and Management of Intermittent Rivers and Ephemeral Streams (\url{ http://www.smires.eu/})
      \item Ist eine EU-COST Action zur Erforschung von Ökosystemen intermittierender Fließgewässer.
      \item Arbeitsgruppe 1 (\textit{Prevalence, distribution and trends of IRES}) soll hydrologische Grundlagen liefern.   
  \end{itemize}
\end{frame}
 
\begin{frame}{Das R Paket \texttt{smires}}{\url{https://github.com/mundl/smires}}
\begin{itemize}
    \item Beinhaltet von jedem Partnerland einen Beispieldatensatz von intermittierenden Flüssen. 
    \item Stellt die gängigen Funktionen zur Berechnung von Niederwasserkenngrößen bereit.
\end{itemize}

\vfill

\textbf{Idee}: Anstatt unzählige (ähnliche) Metriken zu implementieren, Bereitstellung eines Frameworks.

\vfill
\end{frame}

\begin{frame}{Anforderungen}
\begin{itemize}
    \item Richtet sich an R-Anfänger und Fortgeschrittene. 
    \item Einfache Bedienbarkeit
    \item Flexibilität
    \item So wenig Anforderungen wie möglich an die Eingangsdaten (\texttt{NA}, $\Delta t$, ...).
\end{itemize}

\textbf{Idee}: Anstatt unzählige (ähnliche) Metriken zu implementieren, Bereitstellung eines Frameworks.
\end{frame}

\begin{frame}{Demo}
\end{frame}


\end{document}